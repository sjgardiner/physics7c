% Old physhw document class
\documentclass[letterpaper, 12pt]{report}
 
\usepackage{fancyhdr}
\usepackage{graphicx}
\usepackage{empheq}
\usepackage{ifthen}
\usepackage{pgfplots}
\usepackage{booktabs}
\usepackage{caption}
\usepackage[top=1in,bottom=1in,left=1in,right=1in]{geometry}

\newif\ifpgbrks % Conditional that controls whether new problems should
                % start on a new page
\pgbrkstrue     % Default is to start new problems on a new page

\makeatletter

\setlength{\headheight}{15pt}
\lhead{\@author}\chead{\@title}\rhead{\today}
\lfoot{}\cfoot{\thepage}\rfoot{}
\pagestyle{fancy}
 
\ifx\pdfoutput\undefined %LaTeX
\usepackage[ps2pdf,bookmarks=true]{hyperref}
\hypersetup{ %
pdfauthor = {\@author},
pdftitle = {\@title},
pdfcreator = {LaTeX with hyperref package},
pdfproducer = {dvips + ps2pdf}
}
\else %PDFLaTeX
\usepackage[pdftex,bookmarks=true]{hyperref}
\hypersetup{ %
pdfauthor = {\@author},
pdftitle = {\@title},
pdfcreator = {LaTeX with hyperref package},
pdfproducer = {dvips + ps2pdf}
}
\pdfadjustspacing=1
\fi
 
% Set up counters for problems and subsections
 
\newcounter{ProblemNum}
\newcounter{SubProblemNum}[ProblemNum]
 
\renewcommand{\theProblemNum}{\arabic{ProblemNum}}
\renewcommand{\theSubProblemNum}{\alph{SubProblemNum}}
 
\newcommand*{\anyproblem}[1]{\ifpgbrks\newpage\fi\subsection*{#1}}
%\newcommand*{\problem}[1]{\stepcounter{ProblemNum} %
%\anyproblem{Problem \theProblemNum. \; #1}}
%
\newcommand*{\problem}[1][\theProblemNum]{\stepcounter{ProblemNum} %
\anyproblem{Problem #1.}}
%
\newcommand*{\soln}[1]{\subsubsection*{#1}}
\newcommand*{\solution}{\soln{Solution}}
\renewcommand*{\part}{\stepcounter{SubProblemNum} %
\soln{Part (\theSubProblemNum)}}
 
\renewcommand{\theenumi}{(\alph{enumi})}
\renewcommand{\labelenumi}{\theenumi}
\renewcommand{\theenumii}{\roman{enumii}}
 
% Redefines the \maketitle command previously defined by the report document class.
% The default \maketitle macro clears the values of \@author, etc., but we want to keep
% those to print as part of each page's header. To enable the user to add a title page if desired,
% the \maketitle definition from report.cls is repeated here, but with the commands to clear \@author, etc.
% commented out.
\if@titlepage
  \renewcommand\maketitle{\begin{titlepage}%
  \let\footnotesize\small
  \let\footnoterule\relax
  \let \footnote \thanks
  \null\vfil
  \vskip 60\p@
  \begin{center}%
    {\LARGE \@title \par}%
    \vskip 3em%
    {\large
     \lineskip .75em%
      \begin{tabular}[t]{c}%
        \@author
      \end{tabular}\par}%
      \vskip 1.5em%
    {\large \@date \par}%       % Set date in \large size.
  \end{center}\par
  \@thanks
  \vfil\null
  \end{titlepage}%
%  \setcounter{footnote}{0}%
%  \global\let\thanks\relax
%  \global\let\maketitle\relax
%  \global\let\@thanks\@empty
%  \global\let\@author\@empty
%  \global\let\@date\@empty
%  \global\let\@title\@empty
%  \global\let\title\relax
%  \global\let\author\relax
%  \global\let\date\relax
%  \global\let\and\relax
}
\else
\renewcommand\maketitle{\par
  \begingroup
    \renewcommand\thefootnote{\@fnsymbol\c@footnote}%
    \def\@makefnmark{\rlap{\@textsuperscript{\normalfont\@thefnmark}}}%
    \long\def\@makefntext##1{\parindent 1em\noindent
            \hb@xt@1.8em{%
                \hss\@textsuperscript{\normalfont\@thefnmark}}##1}%
    \if@twocolumn
      \ifnum \col@number=\@ne
        \@maketitle
      \else
        \twocolumn[\@maketitle]%
      \fi
    \else
      \newpage
      \global\@topnum\z@   % Prevents figures from going at top of page.
      \@maketitle
    \fi
    \thispagestyle{plain}\@thanks
  \endgroup
%  \setcounter{footnote}{0}%
%  \global\let\thanks\relax
%  \global\let\maketitle\relax
%  \global\let\@maketitle\relax
%  \global\let\@thanks\@empty
%  \global\let\@author\@empty
%  \global\let\@date\@empty
%  \global\let\@title\@empty
%  \global\let\title\relax
%  \global\let\author\relax
%  \global\let\date\relax
%  \global\let\and\relax
}
\fi

\makeatother

%%%%% Old physhw style
\usepackage{amsmath} % AMS Math Package
\usepackage{amssymb} % AMS Math Symbols
\usepackage{amsthm}  % AMS Theorem Formatting
\usepackage{etoolbox} % Needed for defining the \cycle command
\usepackage{booktabs} % Needed for nice-looking tables
\usepackage{caption} % Needed for adding captions to tabular environments, etc.

\setlength\parindent{0pt}

%\pagestyle{empty} % Removes page numbers

\newcommand{\p}[1]{\ensuremath{\left(#1\right)}} % Shortcut for matched variable-size parentheses
\newcommand{\br}[1]{\ensuremath{\left[#1\right]}} % Shortcut for matched variable-size square brackets

% Cycle notation for permutation groups
% Based on code found at http://tex.stackexchange.com/questions/54875/variable-argument-command 
\newcounter{totalcount}
\newcounter{listcount}
\newcommand{\cycle}[1]{
  \ensuremath{
    \setcounter{totalcount}{0} % Reset total count
    \renewcommand*{\do}[1]{\stepcounter{totalcount}} % Reconfigure count
    \docsvlist{#1} % Count number of items
    \setcounter{listcount}{0} % Reset current item count
    \renewcommand*{\do}[1]{\stepcounter{listcount}% Next item
    ##1\ifnum\value{listcount}<\value{totalcount}\;\fi} % Print item
    \left(\docsvlist{#1}\right) % Process list
  }
}

\newcommand{\set}[1]{\ensuremath{\left\{ #1 \right\}}} % for sets written using curly brackets

\newcommand{\abs}[1]{\left| #1 \right|} % for absolute value
\newcommand{\avg}[1]{\left< #1 \right>} % for average

\let\vaccent=\v % rename builtin command \v{} to \vaccent{}
\renewcommand{\v}[1]{\ensuremath{\mathbf{#1}}} % for vectors
\newcommand{\gv}[1]{\ensuremath{\boldsymbol{#1}}}
% for vectors of Greek letters, etc.
\newcommand{\uv}[1]{\ensuremath{\mathbf{\hat{#1}}}} % for unit vector
\newcommand{\guv}[1]{\ensuremath{\boldsymbol{\hat{#1}}}}
% for unit vectors of Greek letters, etc.
\newcommand{\norm}[1]{\left\lVert #1 \right\rVert} % for vector norm
\let\underdot=\d % rename builtin command \d{} to \underdot{}
\renewcommand{\d}[2]{\frac{d #1}{d #2}} % for derivatives
\newcommand{\dd}[2]{\frac{d^2 #1}{d #2^2}} % for double derivatives
\newcommand{\pd}[2]{\frac{\partial #1}{\partial #2}} 
% for partial derivatives
\newcommand{\pdd}[2]{\frac{\partial^2 #1}{\partial #2^2}} 
% for double partial derivatives
\newcommand{\pdc}[3]{\left( \frac{\partial #1}{\partial #2}
 \right)_{#3}} % for thermodynamic partial derivatives
\newcommand{\ket}[1]{\left| #1 \right>} % for Dirac bras
\newcommand{\bra}[1]{\left< #1 \right|} % for Dirac kets
\newcommand{\braket}[2]{\left< #1 \vphantom{#2} \right|
 \left. #2 \vphantom{#1} \right>} % for Dirac brackets
\newcommand{\matrixel}[3]{\left< #1 \vphantom{#2#3} \right|
 #2 \left| #3 \vphantom{#1#2} \right>} % for Dirac matrix elements
\newcommand{\grad}[1]{\gv{\nabla} #1} % for gradient
\let\divsymb=\div % rename builtin command \div to \divsymb
\renewcommand{\div}[1]{\gv{\nabla} \cdot #1} % for divergence
\newcommand{\curl}[1]{\gv{\nabla} \times #1} % for curl
\DeclareMathOperator{\lagr}{\mathcal{L}} % Lagrangian operator written using a curly L
\DeclareMathOperator{\tr}{\mathrm{tr}} % Trace  operation
%\let\baraccent=\= % rename builtin command \= to \baraccent
%\renewcommand{\=}[1]{\stackrel{#1}{=}} % for putting numbers above =
%\newtheorem{prop}{Proposition}
\ifx\c@section\undefined
  % Define the section counter if it is not defined already
  \newcounter{section}
\fi
\newtheorem{thm}{Theorem}[section]
\newtheorem{lem}{Lemma}[]
%\theoremstyle{definition}
%\newtheorem{dfn}{Definition}
%\theoremstyle{remark}
%\newtheorem*{rmk}{Remark}

%%%%%%%%%%%%%%%%%%%%%%%%%%%%%%%%%%%%%%%%%%%%%%%%%%%%%%%%%%%%%%%%%%%%%%%%%%%%%
%%% **** Old problem_template.tex ****
%%%%%%%%%%%%%%%%%%%%%%%%%%%%%%%%%%%%%%%%%%%%%%%%%%%%%%%%%%%%%%%%%%%%%%%%%%%%%
\author{}
\title{Extra Wave Graph Practice}

\begin{document}

%\begin{enumerate}
%\item The 1D harmonic wave function $y(x,t)$ is a
%function of two variables, so to graph the whole thing
%we would need a 3D plot.
%
%\item In Physics 7C, we only look at 2D plots of the
%wave at a specific place or at a specific time.
%
%\item The 1D wave equation has the form
%
%\[y(x,t) = y_0 + A\sin\br{\frac{2\pi t}{T}
%\mp \frac{2\pi x}{\lambda} + \phi}\]
%
%where
%\begin{description}
%\item[$y_0$] is the equilibrium displacement (the 
%value of $y$ exactly in the middle between a peak and a
%trough),
%
%\item[$A$] is the amplitude (on a graph, this is the
%vertical distance from equilibrium to a peak, or,
%equivalently, the vertical distance between a peak and
%a trough divided by two),
%
%\item[$T$] is the period (on a graph of $y$ versus
%\textit{time}, this is the distance between two peaks
%or two troughs),
%
%\item[$\mp$] is a sign that is determined by the
%direction of the wave ($-$ means that the wave is
%moving toward the right [positive $x$ direction],
%while $+$ means the wave is moving toward the
%left [negative $x$ direction]),
%
%\item[$\lambda$] is the wavelength (on a
%graph of $y$ versus \textit{position}, this is the
%distance between two peaks or troughs), and
%
%\item[$\phi$] is the phase constant (determines the
%value of $y$ at time $t = 0$, it's easiest to find this
%piece by plugging in values and solving for it after
%you've found everything else)
%
%\end{description}
%
%\item To write the wave equation using graphs, follow
%these steps:
%
%\begin{enumerate}
%\item Determine where the equilibrium point is.
%This gives you $y_0$.
%\item Read the amplitude, wavelength, and period
%off the graphs using the techniques from the
%description above.
%\item Determine the direction the wave is moving by
%choosing a point and looking at how its displacement
%changes a little bit later in time. You
%will want to choose a point for
%this step that is not at a maximum or a minimum
%initially (otherwise it will go down or go up
%regardless of the direction of the wave).
%\item Find the phase constant by solving the wave
%equation for it using a point with a known displacement
%at a given position and time. If you did all of the
%other steps correctly, any point for which you can get
%a $y$ value from the graph will work, but you will have
%a lot less trouble if you pick a point that is at a
%maximum or a minimum. 
%\end{enumerate}

%\item
\textbf{Practice Problem}:
Determine the wave equation based on the 
graphs below. A solution is given on the next page.

\begin{tikzpicture}
  \begin{axis}[title style={align=center},
               title=$t\:{=}\:##t0##\text{ s}$,
               xmin=0,
               xmax=##xmax##,
               xtick={0,1,...,##xmax##},
               ytick={-##amplitude##,...,##amplitude##},
               minor x tick num={3},
               minor y tick num={1},
               xlabel=x (m),
               ylabel=y (cm),
               width=0.85\textwidth,
               height=0.25\textwidth]
  \addplot[ultra thick,
           domain=0:##xmax+0.5##,samples=200]
        {##amplitude##*sin(deg(2*pi*##t0##/##period##)
         ##spatial_term_sign## deg(2*pi*x/##wavelength##)
         + deg(##phase##))};
  \end{axis}
\end{tikzpicture}

\begin{tikzpicture}
  \begin{axis}[title style={align=center},
               title=$x\:{=}\:##x0##\text{ m}$,
               xmin=0,
               xmax=##tmax##,
               xtick={0,1,...,##tmax##},
               ytick={-##amplitude##,...,##amplitude##},
               minor x tick num={3},
               minor y tick num={1},
               xlabel=t (s),
               ylabel=y (cm),
               width=0.85\textwidth,
               height=0.25\textwidth]
  \addplot[ultra thick,
           domain=0:##tmax+0.5##,samples=200]
  {##amplitude##*sin(deg(2*pi*x/##period##)
       ##spatial_term_sign## deg(2*pi*##x0##/##wavelength##)
       + deg(##phase##))};
  \end{axis}
\end{tikzpicture}

\begin{tikzpicture}
  \begin{axis}[title style={align=center},
               title=$t\:{=}\:##t1##\text{ s}$,
               xmin=0,
               xmax=##xmax##,
               xtick={0,1,...,##xmax##},
               ytick={-##amplitude##,...,##amplitude##},
               minor x tick num={3},
               minor y tick num={1},
               xlabel=x (m),
               ylabel=y (cm),
               width=0.85\textwidth,
               height=0.25\textwidth]
  \addplot[ultra thick,
           domain=0:##xmax+0.5##,samples=200]
        {##amplitude##*sin(deg(2*pi*##t1##/##period##)
         ##spatial_term_sign## deg(2*pi*x/##wavelength##)
         + deg(##phase##))};
  \end{axis}
\end{tikzpicture}

\begin{tikzpicture}
  \begin{axis}[title style={align=center},
               title=$x\:{=}\:##x1##\text{ m}$,
               xmin=0,
               xmax=##tmax##,
               xtick={0,1,...,##tmax##},
               ytick={-##amplitude##,...,##amplitude##},
               minor x tick num={3},
               minor y tick num={1},
               xlabel=t (s),
               ylabel=y (cm),
               width=0.85\textwidth,
               height=0.25\textwidth]
  \addplot[ultra thick,
           domain=0:##tmax+0.5##,samples=200]
  {##amplitude##*sin(deg(2*pi*x/##period##)
       ##spatial_term_sign## deg(2*pi*##x1##/##wavelength##)
       + deg(##phase##))};
  \end{axis}
\end{tikzpicture}

\newpage

%\item
\textbf{Solution}

\[##wave_equation##\]

%\begin{enumerate}
%\item We see from the graphs that $y = 0$ is
%the midpoint between a peak ($y = 1$)
%and a trough ($y = -1$), so it will be the equilibrium
%displacement ($y_0 = 0$).
%\item On the position graph (top graph), the distance
%between two peaks is $6\text{ m} - 3\text{ m}$, so the
%wavelength $\lambda = 3\text{ m}$.
%\item On the time graph (bottom graph), the distance
%between two peaks is $7\text{ s} - 3\text{ s}$, so the
%period $T = 4\text{ s}$.
%\item The peaks are located at $y = 1\text{ cm}$, and
%the troughs are located at $y = -1\text{ cm}$, so the
%amplitude is $A = (1 - (-1))/2 = 1\text{ cm}$.
%\item So far, we have
%
%\[y(x,t) = \br{1\text{ cm}}
%\sin\!\br{\frac{2\,\pi\,t}{4s}
%\mp\frac{2\,\pi\,x}{3m} + \phi}\,.\]
%
%Let's now figure out the direction of the wave. We'll
%need to choose a point on the wave for which we have
%time information. The time graph gives us information
%about $x = 2.25 \text{ cm}$, which is at equilibrium at
%$t = 2\text{ s}$ according to our position graph. Since
%this point is not at a maximum or a minimum on our
%position graph, we will be able to figure out which
%direction the wave is traveling using this point. If it
%were at a maximum on our position graph, it would move
%downward at a slightly later time for both wave
%directions, so we would have to choose a different
%point in such a case.
%
%\newpage
%
%\item At time $t = 2\text{ s}$, the point on the wave
%at $x = 2.25\text{ cm}$ is at equilibrium. If we look
%on the time graph to see where it is a little bit later
%(say, $t = 2.1\text{ s}$), we see that it will move up. 
%Let's try picturing the wave shifted both left and
%right a little to see which direction makes that point
%move up. In the graph below, the black line shows the
%wave at $t = 2\text{ s}$, the blue line shows it
%slightly shifted to the left, and the red line shows it
%slightly shifted to the right. I've drawn a dashed line
%at $x = 2.25\text{ m}$ so that we
%can see more clearly what
%happens to the piece of the wave at that position in
%both cases. Comparing the two possibilities using the
%blue and red waves, we see that, if the wave moves to
%the left, the point at $x = 2.25\text{ m}$ moves up
%like we expect. The wave therefore travels to the left,
%so we pick the $+$ sign to go in front of $2\pi
%x/\lambda$ in our wave equation.
%
%\begin{tikzpicture}
%  \begin{axis}[title style={align=center},
%               title=$t\:{=}\:2s$,
%               xmin=0,
%               xmax=9,
%               xlabel=x (m),
%               ylabel=y (cm),
%               width=0.85\textwidth,
%               height=0.25\textwidth]
%  \addplot[ultra thick,
%           domain=0:9.42478,samples=200]
%        {sin(deg(2*pi*2/4) + deg(2*pi*x/3)
%         + deg(3*pi/2))};
%  \addplot[ultra thick,blue,
%           domain=0:9.42478,samples=200]
%        {sin(deg(2*pi*2.3/4) + deg(2*pi*x/3)
%         + deg(3*pi/2))};
%  \addplot[ultra thick, red,
%           domain=0:9.42478,samples=200]
%        {sin(deg(2*pi*1.7/4) + deg(2*pi*x/3)
%         + deg(3*pi/2))};
%  \addplot[dashed, very thick, domain=-1.2:1.1] (2.5,x);
%  \end{axis}
%\end{tikzpicture}
%
%\item We still need to find the phase constant. For
%this part, it is helpful to pick another point for
%which we can get a $y$ value from the graph. This time,
%we want to choose a point that is at a minimum or
%a maximum.
%Otherwise, when we take the inverse sine ($\sin^{-1}$)
%of both sides of the equation to solve for $\phi$, we
%will end up with two solutions for $\phi$. For example,
%if I choose a point that is at equilibrium (so
%$y(x,t) = 0\text{ cm}$) when I calculate
%$\sin^{-1}(0)$ I get two solutions: $0$ and
%$\pi$. It is possible to figure out the phase constant
%this way, but it's harder, and we might as well choose
%the easy option. Looking back at our position graph for
%$t = 2s$, we see that, at $x = 0\text{ m}$, $y$ is at a
%maximum, so that point will do nicely. From the graph,
%we know then that $y(0\text{ m},2\text{ s}) = 1\text{
%cm}$, and based on our earlier results, we know that
%
%\[y(x,t) = \br{1\text{ cm}}
%\sin\!\br{\frac{2\,\pi\,t}{4s}
%+\frac{2\,\pi\,x}{3m} + \phi}\,,\]
%
%so, plugging in our values of $x$ and $t$, we have
%
%\[y(0\text{ m}, 2\text{ s}) = 
%\br{1\text{ cm}}
%\sin\!\br{\frac{2\,\pi\,(2\text{ s})}{4s} + \phi}
%= \br{1\text{ cm}}\sin\br{\pi + \phi}\,.\]
%
%We found earlier from the graph that 
%
%\[y(0\text{ m},2\text{ s}) = 1\text{
%cm}\,,\]
%
%so, if we set these equations equal to each other, we
%get
%
%\[
%\br{1\text{ cm}}\sin\br{\pi + \phi}
%= 1\text{ cm}\,.\]
%
%We can divide both sides by the amplitude (1 cm). Doing
%so gives us the equation
%
%\[\sin\br{\pi + \phi} = 1\,.\]
%
%Taking the inverse sine of both sides gives us
%
%\[\pi + \phi = \sin^{-1}\br{1} = \frac{\pi}{2}\,.\]
%
%Note that the inverse sine function $\sin^{-1}(z)$
%gives a single value for only two numbers: $z = \pm1$.
%This is why I recommend using a maximum or a minimum
%when solving the wave equation for $\phi$ in this way.  
%Subtracting $\pi$ from both sides gives us
%
%\[\phi = -\frac{\pi}{2}\,.\]
%
%We could leave our answer like this and it would be
%just fine. I usually like to make my phase constants
%positive and in between $0$ and $2\pi$. We can do this
%by adding or subtracting $2\pi$ repeatedly until we
%bring the phase constant within range. This works
%because $2\pi$ represents a full cycle of the sine
%function. If we add $2\pi$ to our result, we get our
%final value
%
%\[\phi = \frac{3\pi}{2}\,.\]
%
%\end{enumerate}
%
%\end{enumerate}

\end{document}
